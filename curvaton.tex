\documentclass[11pt]{article}
\usepackage{amssymb}
\usepackage{amsmath}
\usepackage{float}
\usepackage{epsfig}
\usepackage{braket}
\usepackage{multicol}
%\usepackage{lscape}
\usepackage{enumerate}
\usepackage{multirow}
\DeclareMathOperator{\Tr}{Tr}
\usepackage{array}
\usepackage[top = 1.0in, bottom = 1.0 in, left = 1.0in, right = 1.0in]{geometry}

\def\plotone#1{\centering \leavevmode
\includegraphics[clip=, width=.85\columnwidth]{#1}}
\def\plotoneShrinkSmall#1{\centering \leavevmode
\includegraphics[clip=, width=.50\columnwidth]{#1}}
\def\plotoneShrinkMed#1{\centering \leavevmode
\includegraphics[clip=, width=.55\columnwidth]{#1}}
\def\plotoneShrinkBig#1{\centering \leavevmode
\includegraphics[clip=, width=.65\columnwidth]{#1}}
\def\plottwo#1#2{\centering \leavevmode
\includegraphics[width=.45\columnwidth]{#1} \hfil
\includegraphics[width=.45\columnwidth]{#2}}
\def\plottwoh#1#2{\centering \leavevmode
\includegraphics[height=.32\columnwidth]{#1} \hfil
\includegraphics[height=.32\columnwidth]{#2}}
\def\plottwob#1#2{\centering \leavevmode
\includegraphics[width=.49\columnwidth]{#1} \hfil
\includegraphics[width=.49\columnwidth]{#2}}
\def\plottwor#1#2{\centering \leavevmode
\includegraphics[width=.55\columnwidth,angle=90]{#1} \hfil
\includegraphics[width=.55\columnwidth,angle=90]{#2}}
\def\plotthree#1#2#3{\centering \leavevmode
\includegraphics[width=.3\columnwidth]{#1} \hfil
\includegraphics[width=.3\columnwidth]{#2} \hfil
\includegraphics[width=.3\columnwidth]{#3}}


% Set the margins
%
\setlength{\textheight}{8.5in}
\setlength{\headheight}{.25in}
\setlength{\headsep}{.25in}
\setlength{\topmargin}{0in}
\setlength{\textwidth}{6.5in}
\setlength{\oddsidemargin}{0in}
\setlength{\evensidemargin}{0in}

% Macros
\newcommand{\myN}{\hbox{N\hspace*{-.9em}I\hspace*{.4em}}}
\newcommand{\myZ}{\hbox{Z}^+}
\newcommand{\myR}{\hbox{R}}

\newcommand{\myfunction}[3]
{${#1} : {#2} \rightarrow {#3}$ }



\begin{document}

\author{Luca Victor Iliesiu\\ PI Summer Program}
\date{\today}
\title{Notes on the curvaton model}

\maketitle


We computationally analyse the evolution of perturbations in a generic curvaton model, in which the supplementary curvaton field affects the evolution of the perturbation power-spectra generated by the inflaton. \\
\indent Let us consider that initially the universe is radiation dominated, from the first post-inflationary reheating, with the curvaton field stuck at a value $\sigma(0)$ with the potential of the curvaton given by $V(\sigma) = m_{\sigma}^2 \sigma^2/2$. Furthermore, we initially consider that $H(0) \gg m_{\sigma}$ such that initially the curvaton field will not oscillate. \\
\indent We consider that, initially, both the radiation and dark matter components from the universe come from the inflaton field whose quantum fluctuations are independent of those produced in the curvaton field \footnote{Which means that the particle creation and annihilator operators of the inflaton and the curvaton field commute. }. The value of the Newtonian potential in the Newtonian gauge, with $\phi = \psi$, and the perturbations for the densities, $\delta_r$ and $\delta_{DM}$ and the fluid velocities, $\Theta_r$ and the $\Theta_{DM}$, in the radiation and the dark matter species is given by the adiabatic condition with\footnote{Note that we refer to the radiation, dark matter and dark radiation fluids through the indices $r$, $DM$ and $DR$, while the indices $i$ and $j$ will serve as generic indices that can be used for generalized multi-field models.  }
\begin{eqnarray}
\delta_r = -2 \psi \mbox{ \indent} \delta_{DM} = -\frac{3}2 \delta_{r} \nonumber
\\ \mbox{ \indent} \Theta_r = \Theta_{DM} = \frac{1}2 (k^2 \tau) \psi.
\end{eqnarray}
Furthermore, the ratio of the values density perturbation in the radiation species and the density perturbation in the curvaton field is given by the first-order slow-roll parameter $\epsilon$, from the inflationary period\footnote{This happens since the curvaton energy density and mass is negligible compared to that of the inflaton.}:
\begin{equation} 
\frac{\delta \sigma}{\psi} = -\frac{3}2 \sqrt{\frac{2\epsilon}{4 \pi G}}.
\end{equation}
\indent We evolve these perturbation and follow the curvature perturbations associated with each species  $
\zeta_i = -\psi - H \frac{\delta \rho_i}{\dot \rho_i}
$ and the total curvature perturbation $\zeta = {\sum(\rho_i + p_i) \zeta_i}/{\sum(\rho_i + p_i)}$. As each of the perturbations of the species evolves independently we can characterize the differences between the perturbations in one species and the total perturbation through the value of the iso-curvature $S_{CDM}$ and $S_{DR}$ \footnote{Note that the initial adiabatic condition implies that the initial iso-curvature disappears with,
\begin{equation}
S_{DM}  = 3 \bigg(\frac{\delta_{DM}}{1 + w_{DM}} - \frac{\delta_r}{1 + w_r} \bigg) = 0
\mbox{\indent and \indent}
S_{DR}  = 3 \bigg(\frac{\delta_{DR}}{1 + w_{DR}} - \frac{\delta_r}{1 + w_r} \bigg) = 0
\end{equation}},  
\begin{eqnarray}
S_{CDM} = 3(\zeta_{CDM} - \zeta) \nonumber \\
S_{DR} = 3(\zeta_{DR} - \zeta) 
\end{eqnarray}
\indent Furthermore the independence assumption gives the initial values in the matrix of correlation functions between any two degrees of freedom in the perturbations of all the species, which gives the correlation matrix $P_{\zeta_i \zeta_j}$ between the curvature perturbation associated to each of the species\footnote{Note that by obtaining the correlation matrix $P_{\zeta_i \zeta_j}$ we can obtain the correlation spectra between the total curvature and the iso-curvature of any species $P_{\zeta S_i}$ or between the iso-curvature of any two species $P_{S_i S_j}$.
}. \\
\indent In the case of the curvaton, in order to quantify the way in which the supplementary field affects the power spectra it is useful to identify the following quantities, 
\begin{eqnarray}
\lambda = \frac{P_{\zeta \zeta}(N_{end})}{P_{\zeta \zeta}(0)} - 1 \\
\frac{\alpha_i}{1 + \alpha_i} = \frac{P_{S_i S_i}(N_{end})}{P_{\zeta \zeta}(N_{end})} \\
r_i = \frac{P_{S_i \zeta}(N_{end})}{\sqrt{P_{S_i S_i}(N_{end}) P_{\zeta \zeta}(N_{end})}}\\
r_{ij} = \frac{P_{S_i S_j}(N_{end})}{\sqrt{P_{S_i S_i}(N_{end}) P_{S_j S_j}(N_{end})}}
\end{eqnarray}
where $N_{end}$ is the number of e-folds from the start of the simulation when the curvaton almost completely decays\footnote{Note that the values of $\lambda$, $\alpha_i$ and $r_i$ can be determined at any point by following the evolution of the matrix $P_{\zeta_i \zeta_j}$.}. The parameter $\lambda$ describes how much the power spectra generated by the inflaton is affected by the curvaton decay (for the canonical curvaton $\lambda \rightarrow \infty$), $\alpha_i$ describes how much iso-curvature the curvaton has generated through its decay, while $r_i$ quantifies the correlation between the iso-curvature of the species and the total curvature. \\
\indent In order to interpret the evolution of the perturbations one must understand what parameters we can vary in order to normalize the final power spectrum, after the curvaton has completely decayed, to that which we see in the CMB. We evolve independently, for two separate cosmologies, the set of perturbations that correspond to those generated by the inflaton and those generated by an initial perturbation in the curvaton \footnote{Note that we can use this independence assumption as we describe the evolution of all perturbations to linear order. For instance, we can calculate the two-point power spectra of the total curvature as, 
\begin{equation}
P_{\zeta \zeta} = P_{(\zeta^{inf} + \zeta^{curv}) (\zeta^{inf} + \zeta^{curv}} = P_{\zeta^{inf} \zeta^{inf}} + P_{ \zeta^{curv} \zeta^{curv}}
\end{equation} }
As we vary any of the parameters and normalize the power spectra to that in the CMB we still have a degree of freedom with which we can set either the energy scale of inflation, by setting $H_*$ and varying the slow-roll parameter $\epsilon$, or we can fix the first order slow-roll parameter and change the Hubble parameter during inflation.  \\
\indent 
\begin{figure}

\centering
\includegraphics[width = 3.2in, page=1]{plots/curvCorrelSpectrum.png}
\caption{Evolution of total curvature power spectrum over a number of e-fold during the curvaton decay for different values of $\Gamma$, keeping the energy scale of inflation constant with $H_* = 8.23 \times 10^{11} TeV$. Note that in order to keep the energy scale of inflation constant, we have modify the inflationary slow-roll parameter $\epsilon$ as we vary $\Gamma$. }
\label{fig:powerspectra}
\end{figure}
\indent For some given values of $m_{\sigma}$, $\sigma(0)$ and $\gamma$, we determine the energy scale of inflation by choosing a value of $\lambda $. Afterwards, we vary the value of the three parameters  $m_{\sigma}$, $\sigma(0)$ and $\gamma$ choosing the value of $\epsilon$ such that $H_*$ is fixed, maintaining $A_s = 2.4 \times 10^{-9}$\footnote{Note that by choosing the appropriate value of $\epsilon$, we can obtain any value of $\lambda$. On the other hand, if afterwards we fix the energy scale of inflation and determine $\epsilon$ it is possible to not be able to obtain a physical value of $\epsilon$, with $16\epsilon_* = r < 0.1$ at the pivot scale. }. \\
\indent In Figure \ref{fig:powerspectra} we present the normalized power spectra varying $\Gamma$ choosing $H_*$ such that for the smallest decay value  the value, $\lambda = 5$. For larger decay values $\lambda$ as well as the slow-roll parameter $\epsilon$ decrease\footnote{Note that this happens because $1 + \lambda = \frac{P_{CMB}(2\pi)^2 \epsilon}{H_*}$ which means that $\lambda$ depends linearly on $epsilon$. Furthermore, for a certain value of $\epsilon$ we see that we can obtain a much higher energy level of inflation than by just considering the single-field case.}. Note that using the perturbed Klein-Gordon equation, we cannot make $\Gamma$ take very small values as our simulation has to capture two different time scales -- the time scale of the oscillations of the curvaton field ($\sim 1/m_{\Gamma}$) and the time scale of the decay ($\sim 1/\Gamma$). For this reason during the field oscillations we approximate the evolution of the field as a fluid with $w_\sigma = 0$. \\
\indent In Figure \ref{fig:lambda} we see how the curvaton parameters vary with $\Gamma$ for the dark matter and dark radiation species. For the smallest value of $\Gamma$ we see that the universe is curvaton dominated at the start of the decay while for the highest value the universe is radiation dominated throughout our simulation. As $\lambda$ decreases, we note that the dark matter self iso-curvature power spectra dominates over the curvature power-spectra with $\alpha = 1$ for all the values of $\Gamma$, while in the dark radiation we only see little iso-curvature with $\alpha < 0.2$. On the other hand, we see similar behaviours for the correlation  spectra  between the total curvature and iso-curvature for both the dark matter and dark radiation fluids with $r$ decreasing with $\Gamma$ and going asymptotically towards $1.0$ for small values of $\Gamma$. \\
\indent Finally, in Figure \ref{fig:masssigma} we observe the behaviour of the curvaton parameters (dark matter on the \textit{top} row and dark radiation on the \textit{bottom} row) as we vary $m_{\sigma}$ and $\sigma(0)$. For $\lambda$ we see that highest value is obtained for large $m_{\sigma}$ and $\sigma(0)$ and $m_{\sigma}$ dominates over $\sigma(0)$  \footnote{Note that this is in agreement with the rough estimate of $\lambda$ obtained in \cite{} }. The parameter $\alpha$ is always close to $1.0$ for dark matter with the smallest values for high masses and small values of the field,while in the dark radiation fluid we see only a dependence on  the mass of the curvaton field with the highest values of $\alpha$ excluded by the observations in the CMB. For both the dark matter and dark radiation fluids we see that the parameter $r$ has only a dependence on the mass of the curvaton field.

\begin{figure}
\centering
\includegraphics[width = 3.2in, page=1]{plots/lamalpgam.png} \\
\includegraphics[width = 3.2in, page=1]{plots/rholamalpgam.png} 
\caption{Evolution of curvaton parameters, $\lambda$, $\alpha$ and $r$ as we vary the value of the decay rate $\Gamma$ (\textit{top}). Note that the \textit{blue} plots give the evolution of the dark matter iso-curvature parameters, while the \textit{red} plots give the evolution of the iso-curvature parameters for dark radiation. In the \textit{bottom} plots, for each value of $\Gamma$ we plot the evolution of the energy density of radiation (\textit{blue}), the curvaton field (\textit{red}) and dark matter (\textit{green}).}
\label{fig:lambda}
\end{figure}

\newpage
\begin{multicols}{2}
\end{multicols}
\begin{figure*}
  \centering    
  \includegraphics[width=52mm, trim = 24mm 20mm 35mm 20mm, clip]{plots/DMcontLambdaMassSigma0.png}
  \includegraphics[width=52mm, trim = 24mm 20mm 35mm 20mm, clip]{plots/DMcontAlphaMassSigma0.png}
  \includegraphics[width=52mm, trim = 24mm 20mm 35mm 20mm, clip]{plots/DMcontrMassSigma0.png} \\
    \includegraphics[width=52mm, trim = 24mm 20mm 35mm 20mm, clip]{plots/DRcontLambdaMassSigma0.png}
  \includegraphics[width=52mm, trim = 24mm 20mm 35mm 20mm, clip]{plots/DRcontAlphaMassSigma0.png}
  \includegraphics[width=52mm, trim = 24mm 20mm 35mm 20mm, clip]{plots/DRcontrMassSigma0.png}
  \caption{Dependence of the curvaton parameters as a function of the mass of the field ($m_{\sigma}$) and of the initial field value ($\sigma$). In the \textit{top} we see the dependence of the parameters for dark matter, while in the \textit{bottom} we present the dependence of parameters for dark radiation. } 
\label{fig:masssigma}
\end{figure*}
\begin{multicols}{2}
\end{multicols}
\newpage
\begin{center}
{\Large \textbf{Appendix A: Equations of motion}}
\end{center}
\indent Let us analyse the general case for the evolution of $M$ fields and $P$ fluids in the FRW metric, determining the first order perturbation theory in the Newtonian gauge. Afterwards, in order to consider the curvaton, we will particularize our ODE system, approximating the Klein-Gordon equations for the dynamics of the curvaton and its perturbations in order to eliminate the time scale of its oscillations . \\
\indent We consider an FRW background metric where the evolution of $a(t)$ is given by the Friedmann equation, 
\begin{equation}
H^2 = \bigg(\frac{\dot a}{a} \bigg)^2 =  \frac{8 \pi G}3 \sum_{i \in {1,M} \mbox{and}  {1,P}} \rho_i 
\end{equation}
where we have summed up over all the densities of the scalar fields as well as the fluid densities\footnote{Note that the energy density of all the scalar fields $\phi_1$, $\phi_2$, ...,  $\phi_M$ is given by $\rho_{\phi_1 ... \phi_M} = \sum_{i=1}^M \dot \phi_i^2 + V(\phi_1 ... \phi_M) $. Note that if the potential can be written as $V(\phi_1 ... \phi_M) = \sum_{i=1}^M V(\phi_i)$ we can write a corresponding energy density for each field, otherwise we will only be able to work with the total energy density  $\rho_{\phi_1 ... \phi_M}$. }.\\
\indent The evolution of each scalar field is given by, 
\begin{equation}
\ddot \phi_i + (3H + \Gamma_{\phi_1 ... \phi_M})\dot \phi_i + \frac{\partial V(\phi_1 ... \phi_M)}{\partial \phi_i} = 0 \mbox{ \indent for all scalar fields, } i = \bar{1,M}
\end{equation}
The evolution of the densities for all the species is, 
\begin{equation}
\dot \rho_i + 3(1 + w_i) H = f_i \Gamma_{\phi_1 ... \phi_M}  \rho_{\phi_1 ... \phi_M} \mbox{\indent for all fluids,\ } i = \bar{1,P}
\end{equation}
 where we have summed over all fluids. $\Gamma_{\phi_1 ... \phi_M} $ is a global decay rate for all the fields and due to energy conservation we require that $\sum_{i=1}^P f_i = 1$. Note that the parameter $\Gamma_{\phi_1 ... \phi_M} $  is model dependent and in the specific model of the curvaton we will assume that it is constant -- considering more complicated functions which could lead to phenomena such as parametric resonance in the curvaton scenario are also worth exploring.\\
\indent Now that we have determined the evolution of the background metric, the scalar fields and all the species, we will consider the evolution of their perturbations; for convenience, we will work in the Newtonian gauge, 
\begin{equation}
ds^2 = -(1 + 2\psi)dt^2 + a^2(t)(1 - 2\phi) \delta_{ab} dx^a dx^b
\end{equation}
in which we have eliminated two scalr degrees of freedom of the metric. Furthermore, in an universe without anisotropic stress we find that $\psi = \phi$ .\\
\indent It is also useful, to decompose all the perturbations in Fourier modes, determining the evolution for each mode $\mathbf k$.  The evolution of the perturbation with mode $\mathbf k$ of the scalar field $\phi_i$ is given by the perturbed Klein-Gordon equation, 
\begin{equation}
\delta \ddot \phi_i + (3H + \Gamma_{\phi_1 ... \phi_M})\delta \dot \phi_i  + \frac{k^2}{a^2} \delta \phi_i + \sum_j \frac{\partial ^2 V}{\partial \phi_i \partial \phi_j} \delta \phi_j = -2 \frac{\partial V}{\partial \phi_i} \psi + 4 \dot \phi_i \dot \psi 
\end{equation} 
The density perturbation $\delta_i = \delta\rho_i/\rho_i$ for perturbations in one of the fluids,  
\begin{equation}
\dot \delta_i + (1 + w_i) \frac{\Theta_i}a - 3(1+ w_i) \dot \psi = f_i  \Gamma_{\phi_1 ... \phi_M} \frac{\rho_{\phi_1 ... \phi_M}}{\rho_i}(\delta_{\phi_1 ... \phi_M} - \delta_i + \psi)
\end{equation} 
where $\delta_{\phi_1...\phi_M}$ is the density perturbation in the total energy density of the scalar fields\footnote{Note that $\delta_{\phi_1 ... \phi_M}$ is given by, 
\begin{equation}
\delta_{\phi_1...\phi_M} = \frac{\delta \rho_{\phi_1 ... \phi_M}}{\rho_{\phi_1 ... \phi_M}} = \frac{\sum \dot \phi_i \delta \dot \phi_i + \sum \frac{\partial V}{\partial \phi_i} \delta \phi_i - \sum_i \dot \phi_i^2 \psi}{\sum \dot \phi_i^2 +V(\phi_1...\phi_M)} 
\end{equation} 
}. Furthermore, the evolution of the gradients of the velocities $\Theta_i(\mathbf x, t) = \vec\nabla \mathbf v_i (\mathbf x, t) \longrightarrow \Theta_i(\mathbf k, t) = i\mathbf k \mathbf v_i(\mathbf k, t)$ of each species is determined by, 
\begin{equation}
\dot \Theta_i - \frac{k^2}{a} \bigg(\psi + \frac{3}{4} w_i \delta_i\bigg) + (1 - 3w_i) H \Theta_i = f_i  \Gamma_{\phi_1 ... \phi_M} \frac{\rho_{\phi_1 ... \phi_M}}{\rho_i} \bigg(\frac{1}{1+w_i}\Theta_{\phi_1 ... \phi_M} - \Theta_i \bigg)
\end{equation}
where $ \Theta_{\phi_1...\phi_M}$ is the gradient of the velocity of the scalar field \footnote{Note that for scalar fields the gradient of the velocity can be determined from the stress energy tensor with, \begin{equation}
\Theta_{\phi_1 ... \phi_M} = - \frac{k^2 \delta \phi}{\dot \phi}.
\end{equation}}. \\
\indent The dynamics of the Newtonian potential $\psi$ which we find in the perturbed FRW metric, is determined by, 
\begin{equation}
\dot \psi = -  \frac{4 \pi G a}{3H} \sum \rho_i \delta_i - \bigg(\frac{k^2}{3Ha} + aH\bigg) \psi
\end{equation}
The curvature perturbation associated to each species is defined as the gauge invariant quantity $\zeta_i$, which in the Newtonian gauge can be written as, 
\begin{equation}
\zeta_i = -\psi - H \frac{\delta \rho_i}{\dot \rho_i}
\end{equation}
For radiation, dark matter and neutrino species the curvature perturbation can be written as, 
\begin{equation}
\zeta_i = -\psi + \frac{\delta_i}{3(1+w_i) -  f_i \Gamma_{\phi_1 ... \phi_M} \frac{\rho_{\phi_1 ... \phi_M}}{H \rho_i}} \approx \bigg|_{t\rightarrow \infty} -\psi + \frac{\delta_i}{3(1+w_i)}
\end{equation}
For scalar fields, the associated curvature perturbation is given by rewriting the terms in equation (), 
\begin{equation}
\zeta_{\phi_1 \phi_2 ... \phi_M} = -\psi + \frac{\sum \dot \phi_i \delta \dot \phi_i + \sum \frac{\partial V}{\partial \phi_i} \delta \phi_i - \sum_i \dot \phi_i^2 \psi}{3\bigg(1 +\frac{\Gamma_{\phi_1 \phi_2 ... \phi_M} }{3H} \bigg) \sum_i \dot \phi_i^2}
\end{equation}
\indent It now only remains to particularize our system of equations for the curvaton model in which we consider the evolution of the radiation, dark matter, neutrino and dark radiation species. Let us assume that the curvaton potential can be expanded to second order around its minima with $V(\sigma) = m_{\sigma}^2 \sigma^2/2$ where $m_{\sigma}$ is the mass of the curvaton. Note that in both the Klein-Gordon equations for the background and field perturbations there are two time scales -- the time scale of oscillations ($\sim 1/m_{\sigma}$) and the time scale of decay ($\sim 1/\Gamma_{\sigma}$).\\
\indent As it is practically impossible to follow both time scales as we would have to follow a great number of oscillations ($\sim m_{\sigma}^2/M_{PL}^2 \sim 10^{26} $), in order to be able to simulate the evolution with physical values of $\Gamma_{\sigma}$ we need to eliminate the time-scale of oscillations. In order to do this once the oscillation of the field starts we approximate the evolution of the scalar field as a fluid with the equation of state $w=0$. With such an approximation we can write the evolution of the energy density of the curvaton $\rho_{\sigma}$, the density perturbation $\delta_\sigma$ and the gradient of the speed of the curvaton $\Theta_\sigma$,
\begin{equation}
\dot \rho_{\sigma} + (3H + \Gamma_{\sigma})\rho_\sigma = 0 \mbox{ and }
\end{equation}
\begin{equation}
\dot \delta_{\sigma} +  \frac{\Theta_\sigma}a - 3 \dot \psi = -\Gamma_{\sigma} \psi \mbox{\indent and \indent} \dot \Theta_\sigma - \frac{k^2}{a} \psi + H \Theta_\sigma = 0
\end{equation}
\indent We will thus evolve equation () when the field is not oscillating ($H \gg m_\sigma$) and evolve equation () for later times when the field starts to oscillate and the amplitude is decaying. \\
\indent Furthermore, it is important to note that the decay model we have chosen is generic and it is not required that the curvaton field decay directly to radiation, dark matter and dark radiation with the fractions $f_{r}$, $f_{DM}$ and $f_{DR}$. For instance, one of the most interesting scenarios in the CDM literature, is the case in which the curvaton fields only couples to dark matter and all the $\sigma-$particles decay to dark matter particles. In such a scenario, we assume that the freeze-out temperature $T_{freeze-out}$ of the dark matter species is much greater than the temperature of reheating $T_{reheat}$ at which the $\sigma-$particles decay to the dark matter particles. Thus, the resulting dark matter particles have 
\end{document}
